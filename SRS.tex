\documentclass{article}
\usepackage{CJKutf8}
\usepackage{graphicx}
\usepackage{color}
\begin{document}
\begin{CJK}{UTF8}{bkai}
\title{\Huge \color{blue} 聽歌辨識歌手 }
\author{第六組   李政憲 張哲郡 游登翔 劉彥麟 張友澤}
\maketitle
\begin{figure}[h]
\begin{center}
\includegraphics[width=8.5cm]{sing.jpg}
\end{center}
\label{fig:1}
\end{figure}
\newpage
\section{\huge \bf \color{blue}  introduction\\}

\subsection{\Large Purpose\\}
\large因為當初想說其他組應該都是用抓圖片來做辨識\\ 我們這組就想到那能不能也拿聲音去做辨識呢\\ 就是利用音檔去辨識出我們撥放的歌手是哪位
\\同時間 又會好奇 如果單純是用一首歌(沒有去背景音 )\\
\large能不能也能順利辨識出歌手呢\\

\subsection{\Large Intended Audience and Reading Suggestions\\}
\large 目標族群為常尋找好音樂的人 當它們找到一首好歌時只需要輸入音檔就能知道歌手是誰\\\\\\\\\\\\
\subsection{Project Scope\\}
要能辨識出輸入進去的音檔是我們限制的類別中哪位歌手的\\
 以及要能把音檔轉換成為圖片檔
\newpage


\section{\huge\bf \color{blue}  Overall Description\\}

\subsection{\Large Product Perspective\\}
 \large 這個產品能讓你可能在聽YOUTUBE電台時,聽到在播某首歌曲時就用它來找到是哪位歌手的\\\\\\\\
\subsection{\Large Product Functions}
\large 輸入進去歌曲 就能讓它判斷是我們給的歌手們(分類)中的哪位\\\\\\\\

\subsection{\Large User Classes and Characteristics\\}
  \Large 喜歡聽音樂的人以及類別中的歌手的死忠粉絲\\\\
\newpage
\subsection{\Large Operating Environment \\}
 \Large python3
\newpage

\subsection{\Large Design and Implementation Constraints\\}
   \Large 因為大部分的歌都會有伴奏 有些甚至快大過歌手本身的歌聲,\\
所以實際再辨別時是有難度的
\newpage
\subsection{\Large  Assumptions and Dependencies\\}
  \Large  這份專案是先假設即使沒有去除掉背景音\\
 電腦仍然能透過train來辨別出該歌手的特徵並順利辨識
\newpage


\section{\huge\bf  \color{blue} External Interface Requirements\\}
\subsection{\Large User Interfaces\\}
User Interfaces
\newpage
\subsection{\Large Hardware Interfaces\\}
Hardware Interfaces
\newpage
\subsection{\Large Software Interfaces\\}
 Software Interfaces
\newpage



\section{\huge\bf \color{blue}  System Features }
\subsection{\Large Description and Priority }
 Description and Priority 
\newpage
\subsection{\Large Stimulus/Response Sequences}
Stimulus/Response Sequences
\newpage
\subsection{ \Large Functional Requirements}
 Functional Requirements
\newpage


\section{\huge\bf  \color {blue}  Other Nonfunctional Requirements }
\subsection{ \Large Performance Requirements}
 Performance Requirements
\newpage
\subsection{\Large Safety Requirements }
Safety Requirements
\newpage
\subsection{ \Large Security Requirements }
 Security Requirements
\newpage
\Huge\bf   Thank you for watching


\end{CJK}
\end{document}
