\documentclass{beamer}
\usepackage{xcolor}
\usepackage{graphicx}
\usepackage{CJKutf8}

\usetheme{Madrid}

\setbeamertemplate{footline}
{
\leavevmode
\hbox{
\begin{beamercolorbox}
[wd=1\paperwidth,ht=2.25ex,
dp=1ex,right]{date in head/foot}
\insertframenumber{} /
\inserttotalframenumber\hspace*{2ex}
\end{beamercolorbox}
}\vskip0pt
}

\AtBeginSubsection[]
{
  \begin{frame}<beamer>[shrink]{Outline}
    \tableofcontents[currentsection,currentsubsection]
  \end{frame}
}


\begin{document}
\begin{CJK}{UTF8}{nkai}
\title{開放平台軟體 期末報告}
\author{張友澤  李政憲  游登翔  張哲郡  劉彥麟 }

\begin{frame}
  \titlepage
\end{frame}

\begin{frame}[shrink]{Outline} 
\tableofcontents 
\end{frame}

\section{ Introduction}
\subsection{Introduction to your team}
\subsection{Introduction to the problem you're trying to solve}

\section{Methodology}
\subsection{Input of your model}
\begin{frame}{Methodology}{Input of  model}
	讀入MFCC向量特徵轉換後的.npy壓所檔,\\
	將載入的train data與test data reshape為4個維度,\\
	將train label 與 test label 類別變數轉為one-hot encoding,\\
	即為欲輸入model的f所有資料
 \end{frame}

\subsection{Output of your model}
\begin{frame}{Methodology}{Output of  model}
	每個世代完成後,即輸出一HDF5檔案
\end{frame}

\subsection{Each layer of your model}
\begin{frame}{Methodology}{Each layer of model}

d0 = Input(shape=self.img\_shape)\\
 d1 = conv2d(d0, filters=32, f\_size=2, stride=1, bn=True) 建立卷積層\\
 d2 = maxpooling2d(d1, f\_size=2, stride=2) 建立池化層\\
 d3 = Dropout(0.25)(d2)Dropout層\\
d4 = flatten(d3)  Flatten層\\
d5 = dense(d4, f\_size=128, dr=True, lastLayer=False) 全連接層\\
 d6 = dense(d5, f\_size=5, dr=False, lastLayer=True) 全連接層\\

\end{frame}

\subsection{How you save your model?}
\begin{frame}{Methodology}{How to save  model}
	使用save函式來儲存model至指定資料夾
\end{frame}
\subsection{File size of your model}
\begin{frame}{Methodology}{File size of model}
	每個Model size為2.15 MB
\end{frame}
\subsection{What's your loss functions, and why?}
\begin{frame}{Methodology}{loss functions and why}
	loss function使用'categorical\_crossentropy'\\
	因為用於多個分類,且目標值為分類格式(如:(1,0,0,0,0)、(0,1,0,0,0)),所以選擇採用categorical\_crossentropy作為損失函數
\end{frame}
\subsection{What's your optimizer and the setting of hyperparameter?}
\begin{frame}{Methodology}{optimizer and setting of hyperparameter}
	optimizer採用'Adam'\\
	metrics採用'accuracy'
\end{frame}
\section{Dataset}
\subsection{The size of your dataset should be larger than 1K}
\subsection{How you collect/build your dataset?}
\subsection{How many paired training samples in your dataset?}
\subsection{How many paired validating samples in your dataset?}
\subsection{How many paired testing samples in your dataset?}

\section{Experimental Evaluation}
\subsection{Experimental environment (CPU, GPU, memory,…,etc.)}
\begin{frame}{Experimental environment}
	CPU:\\
	GPU:\\
	RAM:\\
	ROM:\\
\end{frame}
\subsection{How many epochs you set for training?}
\begin{frame}{Experimental environment}{How many epochs set for training}
	99 epochs
\end{frame}
\subsection{Qualitative evaluation}
\begin{frame}{Experimental environment}{Qualitative evaluation}
	
\end{frame}
\subsection{Quantitative evaluation}
\begin{frame}{Experimental environment}{Quantitative evaluation}
	
\end{frame}





\end{CJK}
\end{document}


