\documentclass{beamer}
\usepackage{xcolor}
\usepackage{graphicx}
\usepackage{CJKutf8}

\usetheme{Madrid}

\setbeamertemplate{footline}
{
\leavevmode
\hbox{
\begin{beamercolorbox}
[wd=1\paperwidth,ht=2.25ex,
dp=1ex,right]{date in head/foot}
\insertframenumber{} /
\inserttotalframenumber\hspace*{2ex}
\end{beamercolorbox}
}\vskip0pt
}

\AtBeginSubsection[]
{
  \begin{frame}<beamer>{Outline}
  \tiny{\tableofcontents[currentsection,currentsubsection]}
  \end{frame}
}


\begin{document}
\begin{CJK}{UTF8}{nkai}
\title{開放平台軟體 期末報告}
\author{張友澤  李政憲  游登翔  張哲郡  劉彥麟}

\begin{frame}
  \titlepage
\end{frame}

\begin{frame}{Outline}
\tiny{\tableofcontents}
\end{frame}

\section{Introduction}
\subsection{Introduction to your team}
\begin{frame}{Introduction}{Introduction to your team}
	1051438 張友澤\\
	1051518 李政憲\\ 
	1051540 游登翔\\ 
	1051541 張哲郡\\ 
	1051543 劉彥麟\\
 \end{frame}
\subsection{Introduction to the problem you're trying to solve}
\begin{frame}{Introduction}{Introduction to the problem you're trying to solve}
	有時候在廣播電台中、逛街途中會聽到自己覺得很好聽的音樂,但又不知道是誰唱得的時候,就可以錄一段音樂(要有人聲),再丟進我們的程式,讓他告訴你是誰唱的。
 \end{frame}

\section{Methodology}
\subsection{Input and output of your model}
\begin{frame}{Methodology}{Input and output of model}
	Input:\\
	讀入MFCC向量特徵轉換後的.npy壓所檔,\\
	將載入的train data與test data reshape為4個維度,\\
	將train label 與 test label 類別變數轉為one-hot encoding,\\
	即為欲輸入model的f所有資料\newline\newline
	Output:\\
	每個世代完成後,即輸出一HDF5檔案\\
	
 \end{frame}

\subsection{Each layer of your model}
\begin{frame}{Methodology}{Each layer of model}

d0 = Input(shape=self.img\_shape)\\
 d1 = conv2d(d0, filters=32, f\_size=2, stride=1, bn=True) 建立卷積層\\
 d2 = maxpooling2d(d1, f\_size=2, stride=2) 建立池化層\\
 d3 = Dropout(0.25)(d2) Dropout層\\
d4 = flatten(d3)  Flatten層\\
d5 = dense(d4, f\_size=128, dr=True, lastLayer=False) 全連接層\\
 d6 = dense(d5, f\_size=5, dr=False, lastLayer=True) 全連接層\\

\end{frame}

\subsection{How you save and file size of your model?}
\begin{frame}{Methodology}{How to save  model and file size of model}
	使用save函式來儲存model至指定資料夾\\
	每個Model size為2.15 MB\\
The Fig.\,\ref{fig:2.3.1} 
\begin{figure}
 \includegraphics
 [width=1.0\linewidth]{model_info.jpg} 
\caption{model相關資訊} 
\label{fig:2.3.1}
 \end{figure}

\end{frame}

\subsection{What's your loss functions, and why?}
\begin{frame}{Methodology}{loss functions and why}
	loss function使用'categorical\_crossentropy'\\
	因為用於多個分類,且目標值為分類格式(如:(1,0,0,0,0)、(0,1,0,0,0)),所以選擇採用categorical\_crossentropy作為損失函數
\end{frame}
\subsection{What's your optimizer and the setting of hyperparameter?}
\begin{frame}{Methodology}{optimizer and setting of hyperparameter}
	optimizer採用'Adam'\\
	metrics採用'accuracy'\\
The Fig.\,\ref{fig:2.5.1} 
\begin{figure}
 \includegraphics
 [width=1.0\linewidth]{optimizer.jpg} 
\caption{optimizer and setting of hyperparameter} 
\label{fig:2.5.1}
 \end{figure}
\end{frame}
\section{Dataset}
\subsection {The size of our dataset should be larger than 1K}
\begin{frame}{Dataset}{The size of our dataset should be larger than 1K}
\begin{figure}
\begin{center} 
\includegraphics[height=4cm]{1.png}
\end{center}
\caption{It's our Datasize} 
\end{figure}
\end{frame}
\subsection{How you collect/build  dataset?}
\begin{frame}{Dataset}{How you collect/build  dataset?}
1.把音樂下載成MP3的格式
\newline
\newline
\newline
2.用裁切軟體裁剪成每10秒一個人聲的音訊檔
\newline
\newline
\newline
3.把這些資料取mfcc特徵向量並製作成.npy壓縮檔
\end{frame}
\subsection{How many paired training samples in dataset?}
\begin{frame}{Dataset}{How many paired training samples in  dataset?}
使用此段code,從Dataset中每個類別取160筆資料(總共800筆資料)去訓練成模組
\newline
\newline
\begin{figure}
\begin{center} 
\includegraphics[height=4cm]{SperateCode_TrainData.png}
\end{center}
\caption{利用function把1000筆資料分成800筆}
\end{figure}
\end{frame}
\subsection{How many paired validating samples in dataset?}
\begin{frame}{Dataset}{How many paired validating samples in  dataset?}
使用此段code,從Dataset中每個類別取40筆資料(總共200筆資料,不會與train dataset的資料重複)來驗證模組的準確度
\newline
\newline
\begin{figure}
\begin{center} 
\includegraphics[height=4cm]{SperateCode_ValidatingData.png}
\end{center}
\caption{利用function把1000筆資料分成200筆}
\end{figure}
\end{frame}
\subsection{How many paired testing samples in  dataset?}
\begin{frame}{Dataset}{How many paired testing samples in  dataset?}
總共50筆資料來測試模組
\newline
\newline
\begin{figure}
\begin{center} 
\includegraphics[height=4cm]{6.png}
\end{center}
\caption{50筆隨機歌手人聲的Test Data}
\end{figure}
\end{frame}

\section{Experimental Evaluation}
\subsection{Experimental environment (CPU, GPU, memory,…,etc.) and How many epochs you set for training? }
\begin{frame}{Experimental environment and how many epochs set for training?}
	CPU: Intel i5-4570 3.40GHz\\
	RAM: 16GB\\
	作業系統: Windows 10企業版\\
	系統類型: 64位元作業系統,x64型處理器\\
	Pycharm 2019.1.1 (Professional Edition)\\
	(沒有使用GPU跑model)\newline\newline
	本專題訓練了99個epochs
\end{frame}

\subsection{Qualitative evaluation}
\begin{frame}{Experimental environment}{Qualitative evaluation}
	
\end{frame}
\subsection{Quantitative evaluation}
\begin{frame}{Experimental environment}{Quantitative evaluation}
The Fig.\,\ref{fig:4.3.2} 
\begin{figure}
 \includegraphics
 [width=0.5\linewidth]{model_acc.jpg} 
\caption{每個世代model訓練正確率} 
\label{fig:4.3.2}
 \end{figure}
\end{frame}

\section*{Authorship}
\subsection*{Job scheduling of your team}
\begin{frame}{Authorship}{Job scheduling of your team}
05/31-06/03 每個人上傳自己選擇的歌手的dataset(200個檔案)\\
06/03-06/09 寫codetrain完model到99世代\\
06/09-06/13 Latex建置presentation,SRS
 \end{frame}

\subsection*{Contribution of each team member with evidence}
\begin{frame}{Authorship}{Contribution of each team member with evidence}
張友澤:dataset,SRS\\
李政憲:dataset,Presentation\\ 
游登翔:dataset,Presentation\\ 
張哲郡:dataset,大部分code\\ 
劉彥麟:dataset,SRS\\
 \end{frame}

\end{CJK}
\end{document}


