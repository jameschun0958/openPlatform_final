\documentclass{article}
\usepackage{CJKutf8}
\usepackage{graphicx}
\begin{document}
\begin{CJK}{UTF8}{bkai}
\title{\Huge聽歌辨識歌手 }
\author{第六組   李政憲 張哲郡 游登翔 劉彥麟 張友澤}
\maketitle
\begin{figure}[h]
\begin{center}
\includegraphics[width=8.5cm]{sing.jpg}
\end{center}
\label{fig:1}
\end{figure}

\newpage
\title{\huge\bf  introduction}
\section*{Purpose}
\large因為當初想說其他組應該都是用抓圖片來做辨識\\ 我們這組就想到那能不能也拿聲音去做辨識呢\\ 就是利用音檔去辨識出我們撥放的歌手是哪位
\\同時間 又會好奇 如果單純是用一首歌(沒有去背景音 )\\
能不能也能順利辨識出歌手呢
\section*{Intended Audience and Reading Suggestions}
Here is Section2.
\section*{Project Scope}
Here is Section3.
\newpage


\title{\huge\bf Overall Description}
\section*{Product Perspective}
 Product Perspective
\section*{Product Functions}
輸入進去歌曲 就能讓它判斷是我們給的歌手們(分類)中的哪位
\section*{User Classes and Characteristics}
 User Classes and Characteristics
\section*{Operating Environment }
 Operating Environment 
\section*{Design and Implementation Constraints}
 Design and Implementation Constraints
\section*{ Assumptions and Dependencies}
  Assumptions and Dependencies
\newpage


\title{\huge\bf External Interface Requirements}
\section*{User Interfaces}
User Interfaces
\section*{Hardware Interfaces}
Hardware Interfaces
\section*{Software Interfaces}
 Software Interfaces
\newpage


\title{\huge\bf System Features }
\section*{Description and Priority }
 Description and Priority 
\section*{Stimulus/Response Sequences}
Stimulus/Response Sequences
\section*{ Functional Requirements}
 Functional Requirements
\newpage


\title{\huge\bf Other Nonfunctional Requirements }
\section*{ Performance Requirements}
 Performance Requirements
\section*{Safety Requirements }
Safety Requirements
\section*{ Security Requirements }
 Security Requirements
\newpage


\end{CJK}
\end{document}
